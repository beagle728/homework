\documentclass{article}
\usepackage[utf8]{inputenc}
\usepackage{ctex}
\usepackage{amsmath}
\usepackage{listings}
\usepackage{xcolor}
\usepackage{hyperref}

\title{四则混合运算器设计}
\author{周正颢 3230104251}
\date{2024年12月22日}

\lstset{
    language=C++,
    basicstyle=\ttfamily\small,
    keywordstyle=\color{blue}\bfseries,
    commentstyle=\color{green},
    stringstyle=\color{red},
    showstringspaces=false,
    tabsize=4,
    breaklines=true,
    frame=single,
    numbers=left,
    numberstyle=\tiny\color{gray},
    captionpos=b
}

\begin{document}

    \maketitle

    \section{引言}
    本项目旨在设计并实现一个能够解析和计算四则运算表达式的程序。该程序支持加(+)、减(-)、乘(*)、除(/)和括号(())的中缀表达式,能够判断输入表达式的合法性,并在表达式非法时输出ILLEGAL。此外,程序还支持有限位小数运算,并考虑了负数和科学计数法的输入。

    \section{设计思路}
    \subsection{总体架构}
    程序主要由以下几个部分组成:
    \begin{enumerate}
        \item \textbf{表达式求值器(expression\_evaluator.h)}:负责解析、验证和计算四则运算表达式。
        \item \textbf{主程序(main.cpp)}:处理用户输入、运行测试用例,并与表达式求值器进行交互。
    \end{enumerate}



    \subsection{表达式求值器}
    表达式求值器实现了以下主要功能:
    \begin{itemize}
        \item \textbf{合法性检查}:验证输入表达式是否符合四则运算规则,包括括号匹配、运算符顺序、非法字符检测等。
        \item \textbf{表达式解析与计算}:将中缀表达式转换为后缀表达式(逆波兰表示法),并计算其值。
        \item \textbf{支持负数和科学计数法}:处理一元负号和科学计数法表示的数字。
    \end{itemize}


\subsubsection{合法性规则说明}
为了确保表达式的合法性,制定了以下规则:


\begin{enumerate}
    \item \textbf{运算符和操作数的正确顺序}:
    \begin{itemize}
        \item 表达式不能以除负号(-)外的运算符开头。
        \item 表达式不能以运算符结尾。
        \item 不允许出现连续的运算符,如 `++`、`--`,但允许 `+-` 表示加负数。
    \end{itemize}

    \item \textbf{括号匹配}:
    \begin{itemize}
        \item 每一个左括号 `(` 必须有对应的右括号 `)`。
        \item 括号内的表达式必须是合法的。
    \end{itemize}

    \item \textbf{负数处理}:
    \begin{itemize}
        \item 允许在表达式开头、一对括号后或运算符后使用一元负号(-)。
        \item 不支持一元正号(+),以避免运算符混用导致的歧义。
    \end{itemize}

    \item \textbf{科学计数法}:
    \begin{itemize}
        \item 数字可以使用科学计数法表示,如 `2e2` 表示 `200`。
        \item 科学计数法中的指数部分可以带正号或负号,如 `2e+2` 或 `2e-2`。
        \item 科学计数法必须遵循正确的格式,不能出现连续的指数符号或缺失指数值,如 `1e++2` 或 `1e2e3`。
    \end{itemize}

    \item \textbf{除数为零}:
    \begin{itemize}
        \item 表达式中任何除法运算的除数不能为零,否则视为非法表达式。
    \end{itemize}

    \item \textbf{非法字符}:
    \begin{itemize}
        \item 表达式中仅允许数字、运算符(+、-、*、/)、小数点、括号和科学计数法的 `e` 或 `E`。
        \item 任何其他字符都会导致表达式被标记为非法。
    \end{itemize}
\end{enumerate}

这些规则确保了表达式的合法性,同时支持了负数和科学计数法的输入需求。


    \subsubsection{表达式解析与计算}
    表达式解析与计算采用了Shunting Yard 算法的变体:
    \begin{enumerate}
        \item {解析数字}:识别表达式中的数字,支持小数和科学计数法。
        \item {处理运算符优先级}:使用两个栈分别存储操作数和运算符,根据运算符的优先级决定运算顺序。
        \item {处理括号}:遇到左括号 `(` 时,将其推入运算符栈;遇到右括号 `)` 时,弹出运算符栈中的运算符直到左括号。
        \item {处理一元负号}:将一元负号视为特殊运算符,并赋予更高的优先级,确保其在计算时优先于其他运算符。
        \item {计算结果}:通过弹出运算符栈中的运算符并应用于操作数栈中的数字,最终计算出表达式的结果。
    \end{enumerate}


    \section{测试与结果分析}
    为了验证程序的正确性和鲁棒性,设计了一系列测试用例,涵盖了各种合法和非法的表达式。测试结果如下:

\subsection{测试用例}
\begin{itemize}
    \item \textbf{合法表达式}
    \begin{itemize}
        \item 输入: \(3 + 4 \times 2 / ( 1 - 5 )\)
        \item 输出: \(1\)
        \item 说明: 基本的四则运算,带有括号。
    \end{itemize}
    \item \textbf{括号不匹配}
    \begin{itemize}
        \item 输入: \(3 + 4 \times 2 / ( 1 - 5\)
        \item 输出: ILLEGAL
        \item 说明: 缺少右括号。
    \end{itemize}
    \item \textbf{运算符连续使用}
    \begin{itemize}
        \item 输入: \(3 + * 4\)
        \item 输出: ILLEGAL
        \item 说明: 连续出现 `+` 和 `*`。
    \end{itemize}
    \item \textbf{表达式以运算符开头}
    \begin{itemize}
        \item 输入: \(+ 3 + 4\)
        \item 输出: ILLEGAL
        \item 说明: 表达式以 `+` 开头,不允许一元正号。
    \end{itemize}
    \item \textbf{表达式以运算符结尾}
    \begin{itemize}
        \item 输入: \(3 + 4 +\)
        \item 输出: ILLEGAL
        \item 说明: 表达式以 `+` 结尾。
    \end{itemize}
    \item \textbf{除数为0}
    \begin{itemize}
        \item 输入: \(3 / 0\)
        \item 输出: ILLEGAL
        \item 说明: 除数为零。
    \end{itemize}
    \item \textbf{包含非法字符}
    \begin{itemize}
        \item 输入: \(3 + 4 @ 2\)
        \item 输出: ILLEGAL
        \item 说明: 包含非法字符 `@`。
    \end{itemize}
    \item \textbf{小数运算}
    \begin{itemize}
        \item 输入: \(3.5 + 2.5 \times 4.0\)
        \item 输出: \(13.5\)
        \item 说明: 含有小数点的运算。
    \end{itemize}
    \item \textbf{多重括号}
    \begin{itemize}
        \item 输入: \(( ( 3 + 4 ) \times ( 2 - 1 ) )\)
        \item 输出: \(7\)
        \item 说明: 多重嵌套括号。
    \end{itemize}
    \item \textbf{空表达式}
    \begin{itemize}
        \item 输入: \(\)
        \item 输出: ILLEGAL
        \item 说明: 空表达式。
    \end{itemize}
    \item \textbf{单个数字}
    \begin{itemize}
        \item 输入: \(42\)
        \item 输出: \(42\)
        \item 说明: 单个数字。
    \end{itemize}
    \item \textbf{负数}
    \begin{itemize}
        \item 输入: \(1 + -2.1\)
        \item 输出: \(-1.1\)
        \item 说明: 加负数。
    \end{itemize}
    \item \textbf{科学计数法}
    \begin{itemize}
        \item 输入: \(-1 + 2e2\)
        \item 输出: \(199\)
        \item 说明: 负数和科学计数法结合。
    \end{itemize}
    \item \textbf{负数和科学计数法结合}
    \begin{itemize}
        \item 输入: \(1 + -2.1e-1\)
        \item 输出: \(0.79\)
        \item 说明: 加负的科学计数法数值。
    \end{itemize}
    \item \textbf{复杂的科学计数法}
    \begin{itemize}
        \item 输入: \(1.23e-2 + 4.56e3\)
        \item 输出: \(4560.0123\)
        \item 说明: 含有多个科学计数法数字。
    \end{itemize}
    \item \textbf{科学计数法中的正号}
    \begin{itemize}
        \item 输入: \(1e+2 + 2\)
        \item 输出: \(102\)
        \item 说明: 科学计数法中指数带正号。
    \end{itemize}
    \item \textbf{科学计数法中的负号}
    \begin{itemize}
        \item 输入: \(-1e+2 + 2\)
        \item 输出: \(-98\)
        \item 说明: 负数和科学计数法结合。
    \end{itemize}
    \item \textbf{非法的科学计数法}
    \begin{itemize}
        \item 输入: \(1e++2\)
        \item 输出: ILLEGAL
        \item 说明: 连续的指数符号。
    \end{itemize}
    \item \textbf{连续的科学计数符号}
    \begin{itemize}
        \item 输入: \(1e2e3\)
        \item 输出: ILLEGAL
        \item 说明: 连续的科学计数符号。
    \end{itemize}
    \item \textbf{多个小数点}
    \begin{itemize}
        \item 输入: \(1.2.3 + 4\)
        \item 输出: ILLEGAL
        \item 说明: 数字中含有多个小数点。
    \end{itemize}
    \item \textbf{复杂嵌套括号}
    \begin{itemize}
        \item 输入: \(((1 + 2) \times (3 + (4 - 5)))\)
        \item 输出: \(6\)
        \item 说明: 深度嵌套的括号。
    \end{itemize}
    \item \textbf{运算符混用}
    \begin{itemize}
        \item 输入: \(-3 + +4\)
        \item 输出: ILLEGAL
        \item 说明: 运算符混用,不允许一元正号。
    \end{itemize}
    \item \textbf{多个一元运算符}
    \begin{itemize}
        \item 输入: \(-3 + -4\)
        \item 输出: \(-7\)
        \item 说明: 连续的一元负号。
    \end{itemize}
    \item \textbf{不支持的运算符}
    \begin{itemize}
        \item 输入: \(3 + 4 \times -2 / (1 - 5)\^2\)
        \item 输出: ILLEGAL
        \item 说明: 运算符 \^ 不被支持。
    \end{itemize}
    \item \textbf{不完整的科学计数法}
    \begin{itemize}
        \item 输入: \(2e\)
        \item 输出: ILLEGAL
        \item 说明: 缺少指数值。
    \end{itemize}
    \item \textbf{科学计数法后缺符号}
    \begin{itemize}
        \item 输入: \(2e+\)
        \item 输出: ILLEGAL
        \item 说明: 指数符号后缺少数字。
    \end{itemize}
    \item \textbf{括号不匹配 - 多余右括号}
    \begin{itemize}
        \item 输入: \((1 + 2))\)
        \item 输出: ILLEGAL
        \item 说明: 多余的右括号。
    \end{itemize}
    \item \textbf{括号不匹配 - 缺少右括号}
    \begin{itemize}
        \item 输入: \(((1 + 2)\)
        \item 输出: ILLEGAL
        \item 说明: 缺少右括号。
    \end{itemize}
    \item \textbf{一元负号与括号结合}
    \begin{itemize}
        \item 输入: \(-(5+6)\)
        \item 输出: \(-11\)
        \item 说明: 正确处理一元负号与括号的结合。
    \end{itemize}
    \item \textbf{非法的连续运算符}
    \begin{itemize}
        \item 输入: \(1++2.1\)
        \item 输出: ILLEGAL
        \item 说明: 连续的加号运算符。
    \end{itemize}
\end{itemize}

    \subsection{结果分析}
    通过对上述测试用例的验证,得出以下结论:
    \begin{itemize}
        \item \textbf{正确性}:程序能够准确地计算合法表达式的结果,支持加、减、乘、除、括号、负数和科学计数法。
        \item \textbf{合法性检测}:程序能够有效检测和标记非法表达式,如括号不匹配、运算符连续使用、表达式以运算符开头或结尾、除数为零、非法字符和科学计数法使用错误等。
        \item \textbf{鲁棒性}:程序在面对各种边界情况和异常输入时表现出良好的鲁棒性,能够稳定运行并给出正确的反馈。
    \end{itemize}

\end{document}
