\documentclass[UTF8]{ctexart}
\usepackage{geometry}
\geometry{margin=1.5cm, vmargin={0pt,1cm}}
\setlength{\topmargin}{-1cm}
\setlength{\paperheight}{29.7cm}
\setlength{\textheight}{25.3cm}

% useful packages.
\usepackage{amsfonts}
\usepackage{amsmath}
\usepackage{amssymb}
\usepackage{amsthm}
\usepackage{enumitem} 
\usepackage{graphicx}
\usepackage{multicol}
\usepackage{fancyhdr}
\usepackage{layout}
\usepackage{listings}
\usepackage{float, caption}

\lstset{
    basicstyle=\ttfamily, basewidth=0.5em
}

% some common command
\newcommand{\dif}{\mathrm{d}}
\newcommand{\avg}[1]{\left\langle #1 \right\rangle}
\newcommand{\difFrac}[2]{\frac{\dif #1}{\dif #2}}
\newcommand{\pdfFrac}[2]{\frac{\partial #1}{\partial #2}}
\newcommand{\OFL}{\mathrm{OFL}}
\newcommand{\UFL}{\mathrm{UFL}}
\newcommand{\fl}{\mathrm{fl}}
\newcommand{\op}{\odot}
\newcommand{\Eabs}{E_{\mathrm{abs}}}
\newcommand{\Erel}{E_{\mathrm{rel}}}

\begin{document}

\pagestyle{fancy}
\fancyhead{}
\lhead{周正颢, 3230104251}
\chead{数据结构与算法第四次作业}
\rhead{Oct.20th, 2024}

\section{测试程序设计思路}

为了验证 \texttt{List} 类的正确性,我设计了以下几个测试场景:

\begin{enumerate}[label=\arabic*.]
    \item **构造函数测试**:使用默认构造函数和初始化构造函数创建链表对象,检查链表的初始状态。
    \item **插入操作测试**:使用 \texttt{push\_back} 和 \texttt{push\_front} 方法向链表中插入元素,确保元素能够正确添加到链表的两端。
    \item **删除操作测试**:使用 \texttt{pop\_back} 和 \texttt{pop\_front} 方法从链表中删除元素,检查删除后链表的状态和大小是否正确,并尝试从空链表中调用这些方法以触发 bug。
    \item **迭代器测试**:通过迭代器遍历链表,验证是否能够正确访问所有元素。
    \item **边界条件测试**:测试空链表和只有一个元素的链表,确保操作不会引发异常。
\end{enumerate}

\section{测试结果}

经过全面测试,测试结果显示大部分功能正常,具体如下:

\begin{itemize}
    \item 构造函数能够成功创建链表,并在初始化后状态正常。
    \item 插入和删除操作后,链表的大小和内容均符合预期。不过,在测试 \texttt{pop\_back} 方法从空链表中调用时,程序崩溃,表明对空状态的处理不足。
    \item 迭代器能够顺利遍历链表中的所有元素,结果与插入的元素一致。
    \item 使用 \texttt{valgrind} 工具进行内存检测,确认没有内存泄露或未释放的内存。
\end{itemize}

\end{document}

%%% Local Variables: 
%%% mode: latex
%%% TeX-master: t
%%% End: 
