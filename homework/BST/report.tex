\documentclass[UTF8]{ctexart}
\usepackage{geometry}
\geometry{margin=1.5cm, vmargin={0pt,1cm}}
\setlength{\topmargin}{-1cm}
\setlength{\paperheight}{29.7cm}
\setlength{\textheight}{25.3cm}

% useful packages.
\usepackage{amsfonts}
\usepackage{amsmath}
\usepackage{amssymb}
\usepackage{amsthm}
\usepackage{enumitem} % 使用 enumitem 包代替 enumerate
\usepackage{graphicx}
\usepackage{multicol}
\usepackage{fancyhdr}
\usepackage{layout}
\usepackage{listings}
\usepackage{float, caption}

\lstset{
    basicstyle=\ttfamily, basewidth=0.5em
}

% some common command
\newcommand{\dif}{\mathrm{d}}
\newcommand{\avg}[1]{\left\langle #1 \right\rangle}
\newcommand{\difFrac}[2]{\frac{\dif #1}{\dif #2}}
\newcommand{\pdfFrac}[2]{\frac{\partial #1}{\partial #2}}
\newcommand{\OFL}{\mathrm{OFL}}
\newcommand{\UFL}{\mathrm{UFL}}
\newcommand{\fl}{\mathrm{fl}}
\newcommand{\op}{\odot}
\newcommand{\Eabs}{E_{\mathrm{abs}}}
\newcommand{\Erel}{E_{\mathrm{rel}}}

\begin{document}

\pagestyle{fancy}
\fancyhead{}
\lhead{周正颢, 3230104251}
\chead{数据结构与算法第5次作业}
\rhead{Nov.4th, 2024}

\section{对修改后remove函数实现的阐述}

在本次作业中,我对 remove 函数进行了修改,提高删除节点的效率和正确性。新函数的实现逻辑如下:

1. 查找目标节点: 首先,我们递归遍历树,查找要删除的节点。如果找到目标节点,则根据其子节点的情况决定删除方式。

2. 节点删除:
   情况一: 如果目标节点有两个子节点,使用右子树中的最小节点替换目标节点的值,然后删除该最小节点。
   情况二: 如果目标节点只有一个子节点,直接将该子节点替换目标节点。
   情况三: 如果目标节点没有子节点,直接删除目标节点。

通过这种方式,我确保了树的结构在删除节点后仍然保持为二叉搜索树的性质,并且优化了删除过程中的遍历时间。

\section{对测试输出结果的呈现和分析}
在测试中,我使用了一系列操作验证了remove函数的正确性。初始树插入的节点为 11, 8, 6, 3, 7, 12, 4。执行删除操作后,首先删除了节点7。接着,我又删除了节点8。在所有操作完成后,我还进行了树的清空操作,验证了makeEmpty函数是否能正确清空树。最终的测试输出结果符合预期,没有出现异常,验证了remove函数的正确性和效率。

\end{document}

%%% Local Variables: 
%%% mode: latex
%%% TeX-master: t
%%% End: 
